\documentclass[11pt, oneside]{article}   	% use "amsart" instead of "article" for AMSLaTeX format
%\usepackage{geometry}                		% See geometry.pdf to learn the layout options. There are lots.
%\geometry{letterpaper}                   		% ... or a4paper or a5paper or ... 
%\geometry{landscape}                		% Activate for rotated page geometry
%\usepackage[parfill]{parskip}    		% Activate to begin paragraphs with an empty line rather than an indent

\usepackage{geometry}
 \geometry{
 a4paper,
 total={170mm,257mm},
 left=20mm,
 top=15mm,
 }

\usepackage{graphicx}				% Use pdf, png, jpg, or eps§ with pdflatex; use eps in DVI mode
								% TeX will automatically convert eps --> pdf in pdflatex		
\usepackage{amssymb}
\usepackage{amsmath}
\usepackage{fancyhdr}
\usepackage[utf8]{inputenc}
\usepackage[english]{babel}
\usepackage{enumerate}
\usepackage{arcs}
\usepackage{cancel}
\usepackage{xfrac}
\usepackage{tikz}

%SetFonts

%SetFonts

\usepackage[inline]{asymptote}


\pagestyle{fancy}
\fancyhf{}
%\rhead{Teacher David @ 18601688612}
\lhead{\leftmark}


\title{Math Notes}
\author{Peter He}
%\date{}							% Activate to display a given date or no date

\begin{document}
\maketitle




\section{Linear Functions}
\subsection{$y=kx$}
1. $k\ne0$\\
2. The line passes through the fixed point $(0,0)$\\
3. $k$ is the gradient of the line\\
4. If the line passes through another point $(a,b)$, then $k=\frac{b-0}{a-0}=\frac{b}{a}$\\


\begin{center}
\begin{tikzpicture}[scale=0.7]

\draw[blue] (-1,-2) -- (2,4);
\draw[->,thick] (-3,0) -- (3,0);
\draw[->,thick] (0,-3) -- (0,3);
\filldraw(1,2)circle(1pt);

\coordinate [label=below:$x$] (x) at (3,0);
\coordinate [label=left:$y$] (y) at (0,3);
\coordinate [label=right:{$(a, b)$}] (a) at (1,2);


\end{tikzpicture}
\end{center}



\subsection{$y=kx+b$}











\renewcommand{\labelenumii}{(\arabic{enumii})}


\begin{enumerate}
\renewcommand{\labelenumi}{1.\arabic{enumi}}

\item \label{rule:add} \emph{Addition Principle}. If there are $a$ varieties of soup and $b$ varieties of salad, then there are $a+b$ possible ways to order a meal of soup \emph{or} salad (but not both soup and salad). 

\end{enumerate}




\end{document} 