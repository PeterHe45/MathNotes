\documentclass[11pt, oneside]{article}   	% use "amsart" instead of "article" for AMSLaTeX format
%\usepackage{geometry}                		% See geometry.pdf to learn the layout options. There are lots.
%\geometry{letterpaper}                   		% ... or a4paper or a5paper or ... 
%\geometry{landscape}                		% Activate for rotated page geometry
%\usepackage[parfill]{parskip}    		% Activate to begin paragraphs with an empty line rather than an indent

\usepackage{geometry}
 \geometry{
 a4paper,
 total={170mm,257mm},
 left=20mm,
 top=15mm,
 }

\usepackage{graphicx}				% Use pdf, png, jpg, or eps§ with pdflatex; use eps in DVI mode
								% TeX will automatically convert eps --> pdf in pdflatex		
\usepackage{amssymb}
\usepackage{amsmath}
\usepackage{fancyhdr}
\usepackage[utf8]{inputenc}
\usepackage[english]{babel}
\usepackage{enumerate}
\usepackage{arcs}
\usepackage{cancel}
\usepackage{xfrac}
\usepackage{tikz}

%SetFonts

%SetFonts

\usepackage[inline]{asymptote}


\pagestyle{fancy}
\fancyhf{}
%\rhead{Teacher David @ 18601688612}
\lhead{\leftmark}


\title{Math Notes}
\author{Peter He}
%\date{}							% Activate to display a given date or no date

\begin{document}
\maketitle




\section{Linear Functions}
\subsection{$y=kx$}
\begin{enumerate}
%\renewcommand{\labelenumi}{1.\arabic{enumi}}

\item $k\ne0$.
\item The line passes through the fixed point $(0,0)$.
\item $k$ is the gradient of the line.
\item If the line passes through another point $(a,b)$, then $k=\frac{b-0}{a-0}=\frac{b}{a}$, and therfore, $y=\frac{b}{a} x$.
\item If we move the line along the $y$-axis up for $b$ units, we will get another function $y=kx+b$. 
\end{enumerate}

\begin{center}
\begin{tikzpicture}[scale=0.7]

\draw[blue] (-1,-2) -- (2,4);
\draw[red] (-1,0) -- (2,6);
\draw[->,thick] (-3,0) -- (3,0);
\draw[->,thick] (0,-3) -- (0,3);
\filldraw(1,2)circle(1pt);

\coordinate [label=below:$x$] (x) at (3,0);
\coordinate [label=left:$y$] (y) at (0,3);
\coordinate [label=right:{$(a, b)$}] (a) at (1,2);
\coordinate [label=below:{$y=kx$}] (k) at (3.5,4);
\coordinate [label=below:{$y=kx+b$}] (l) at (3.5,6);


\end{tikzpicture}
\end{center}



\subsection{$y=kx+b$}
\begin{enumerate}


\item If $k>0$, the line is increasing; if $k<0$, the line is decreasing.
\item The line intersects with the $y$-axis at point $(0,b)$.
\item Point Slope Form: If the slope is $k$, and the line passes through point $(x_0,y_0)$, then \[y=k(x-x_0)+y_0\].
\item Any two distinctive points $(x_1,y_1), (x_2,y_2)$ determine a line, where $k=\frac{y_2-y_1}{x_2-x_1}$ and \[y=\frac{y_2-y_1}{x_2-x_1}(x-x_1)+y_1\] or  \[y=\frac{y_2-y_1}{x_2-x_1}(x-x_2)+y_2\]
\end{enumerate}


\section{Quadratic Function}
Given Quadratic Function: $y=ax^2+bx+c, a \ne 0$
we can turn the Function into the Vertex Form.
\begin{align*}
y&=ax^2+bx+c\\
&=a\left(x^2+\frac{b}{a}x+\frac{c}{a}\right)\\
&=a\left[(x+\frac{b}{2a})^2-\frac{b^2}{4a^2}+\frac{4ac}{4a^2}\right]\\
&=a\left(x+\frac{b}{2a}\right)^2+\frac{4ac-b^2}{4a}
\end{align*}











\end{document} 